\documentclass[hidelinks,12pt]{article}

\usepackage{amssymb,amsmath,amsfonts,eurosym,geometry,ulem,graphicx,caption,color,setspace,sectsty,comment,footmisc,caption,natbib,pdflscape,array,hyperref}
\usepackage[en-US]{datetime2}
\usepackage{booktabs}
\usepackage{longtable} 
\usepackage{enumitem}
\usepackage{graphicx}
%\usepackage{subfig}
\usepackage{endnotes}
\let\footnote=\endnote
% WORKING
\usepackage{caption, subcaption} 
\usepackage[nomarkers,nolists]{endfloat}

\usepackage{dsfont}
\normalem
\DeclareMathOperator*{\argmin}{arg\,min}

\usepackage{bm}
\usepackage[title]{appendix}

\usepackage{booktabs, fixltx2e}
\usepackage[flushleft]{threeparttable}


\onehalfspacing
\newtheorem{theorem}{Theorem}
\newtheorem{corollary}[theorem]{Corollary}
\newtheorem{proposition}{Proposition}
\newenvironment{proof}[1][Proof]{\noindent\textbf{#1.} }{\ \rule{0.5em}{0.5em}}

\newtheorem{hyp}{Hypothesis}
\newtheorem{subhyp}{Hypothesis}[hyp]
\renewcommand{\thesubhyp}{\thehyp\alph{subhyp}}

\newcommand{\red}[1]{{\color{red} #1}}
\newcommand{\blue}[1]{{\color{blue} #1}}

\newcolumntype{L}[1]{>{\raggedright\let\newline\\arraybackslash\hspace{0pt}}m{#1}}
\newcolumntype{C}[1]{>{\centering\let\newline\\arraybackslash\hspace{0pt}}m{#1}}
\newcolumntype{R}[1]{>{\raggedleft\let\newline\\arraybackslash\hspace{0pt}}m{#1}}

\geometry{left=1.0in,right=1.0in,top=1.0in,bottom=1.0in}
\newcommand\cites[1]{\citeauthor{#1}'s\ (\citeyear{#1})}

\usepackage{mathtools, nccmath}

\begin{document}

%\begin{titlepage}
\title{Meeting Notes on Environmental Issues Related to Agriculture }

\author{Raghav Goyal and Siddhartha Bora}


\date{}
\maketitle
\doublespacing

\subsection*{Meeting Notes, April 1, 2022}
\begin{itemize}
    \item Possible topics:
    \begin{enumerate}
        \item Agriculture and Climate Change
        \item Potential use of GIS data
        \item Agricultural Policy issues relevant to Farm Bill: Cover crops, CRP etc. 
    \end{enumerate}
    \item Timeline:
    \begin{enumerate}
        \item Generate an idea and do some analysis by summer.
        \item Working paper by Fall 2022. 
    \end{enumerate}
    \item Project Housekeeping: 
    \begin{enumerate}
        \item Keep this note file updated after we meet for future reference, and for sharing ideas.
        \item Do all the Overleaf writing inside the ``2\_notes" subfolder, and commit to github from Overleaf.
        \item The ``2\_notes/references.bib" file is synced with a Mendeley group. Create a Mendeley account and upload the related papers to that group. We can read and annotate the references there. I think you don't have to link Mendeley to Overleaf for this project as it is already synced through my account. I have to add you in Mendeley.  
        \item Do all data cleaning within ``0\_data". If you are analysing large raw data files in your local machine, keep it inside ``0\_data/raw", which is ign ored for version control. Only small, cleaned dataset may be kept inside ``0\_data/clean" for shared analysis, if the data sharing policies of the relevant dataset allows uploading to third party websites.
        \item Similarly, do all analysis within ``1\_analysis". Use relative file paths to access figures and tables from ``2\_notes".
        \item The repository is created through R project of same name, so it will work with R seamlessly. But we can keep any other tools.
    \end{enumerate}
\end{itemize}



\pagebreak
\begin{singlespace}
\setlength\bibsep{0pt}
\bibliographystyle{chicago}
\bibliography{references.bib}
\end{singlespace}
\end{document}
